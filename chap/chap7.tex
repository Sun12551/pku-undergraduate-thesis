% Copyright (c) 2014,2016,2018 Casper Ti. Vector
% Public domain.

\chapter{结论与展望}
%\pkuthssffaq % 中文测试文字。
这项研究对使用 Rust 语言编写的操作系统 Asterinas 中文件系统的实现进行了完善,新增了对 exFAT 文件系统的支持。
同时,我们还改进了 Asterinas 中文件系统页缓存部分的实现,使其能够支持数据预取功能。
这一改进不仅可以提升 exFAT 的性能,还可以应用于 Asterinas 中所有使用页缓存的文件系统。

为了验证实现的正确性,我们为新增的 exFAT 文件系统编写了一系列单元测试。
每个测试样例对文件系统的某个功能进行检测,在检测基本功能的同时考虑到了各种边角情况。
为了增加测试的强度,我们又设计了一个自动化随机测试单元,可以自动生成一系列合法的操作并自动检测正确性。
这个测试单元不仅可以用于 exFAT 的测试,还可以用于检测未来 Asterinas 中开发的其他文件系统的正确性。
测试结果显示,我们实现的 exFAT 文件系统成功地通过了所有的单元测试和集成测试,证明了我们的实现是正确的。

在性能测试方面,我们利用 FIO 工具进行了文件 IO 测试,结果基本符合预期。
在数据预取引入前,是否使用页缓存对各种读写模式的性能没有明显影响。
这是因为初次进行文件读写需要对页缓存进行初始化,性能瓶颈在与块设备的交互上。
引入数据预取后,顺序读写模式的性能得到了明显的提高,同时随机读写模式的性能没有受到明显的影响。
数据预取的引入使得在顺序读写模式下,文件可以以更大的块粒度被缓存进内存。
IO 带宽可以更快地达到峰值,进而提升了文件系统在顺序读写模式下的表现。

最后,在不使用页缓存的情况下,文件系统将直接与底层块设备进行交互。
测试结果显示,Asterinas 与块设备的交互的性能与 Linux 还存在较大的差距。
分析表明,这可能是因为被测试的 Asterinas 中块设备层的实现尚不支持异步 IO,所有的读写都是以同步方式进行。
这可能是目前的 Asterinas 中 exFAT 文件系统与 Linux 中性能存在差距的主要原因,也是下一步改进的主要方向。

总的来说,Asterinas 操作系统的开发对于提升操作系统的安全性具有重要的意义,在信息安全越来越受到关注的当下更是如此。
本研究完善了其文件系统的实现,对于 Asterinas 的早日投入生产使用具有重要的意义,所以具有较高的实用价值。
但性能测试的结果表明,我们实现的文件系统的性能还有很大的提升空间。
综合来看,Asterinas 操作系统的开发工作任重道远,需要在性能优化、安全性加固、稳定性测试、用户体验等方面持续努力,
才能最终打造出一个高可靠性、高性能的实用操作系统,为用户提供更安全、更可靠的计算环境。

% vim:ts=4:sw=4
