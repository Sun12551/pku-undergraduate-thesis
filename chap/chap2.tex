% Copyright (c) 2014,2016 Casper Ti. Vector
% Public domain.

\chapter{背景}
%\pkuthssffaq % 中文测试文字。
\section{exFTA文件系统}
exFAT (Extended File Allocation Table)\parencite{exFAT} 是由微软公司开发的一种文件系统,
专为闪存存储,如USB闪存驱动器、SD卡和CF卡等设计。作为FAT32的后继文件系统,
exFAT主要解决了FAT32在处理大文件和大容量存储设备上的限制。具体来说,
exFAT文件系统使用64位来描述文件大小,从而支持依赖于非常大的文件的应用程序;
exFAT文件系统还允许最大32MB的簇(Cluster,硬盘上逻辑存储的基本单位),有效地支持了非常大的存储设备。

作为FAT文件系统家族中的一员,exFAT也使用文件分配表(FAT,File Allocation Table)
来组织文件。exFAT将存储设备划分为一系列的簇,使用文件分配表来描述文件系统中簇的分配状态
以及簇和文件内容之间链接关系。这种组织结构类似链表,文件分配表中维护了当前簇的下一个簇的编号。
在读取某个文件时,通过查找文件分配表可以依次找到文件在磁盘上对应的各个簇的编号,从而访问文件的内容。

exFAT文件系统主要包括超级块、文件分配表、根目录、bitmap、大写转换表、数据区等部分组成。
超级块中包含文件系统的基本信息,如簇的大小、文件分配表和根目录的位置。
超级块的信息会在exFAT打开时被读取,利用超级块中的信息,操作系统可以读取文件分配表并初始化根目录。
bitmap和大写转换表都是根目录下的特殊系统文件,其中bitmap用于跟踪磁盘上的空闲簇和已分配簇,
大写转换表用于支持exFAT对大小写不敏感的特性。数据区则是用于存储用户的文件和文件夹的区域,
同样用文件分配表进行文件组织。第三章会详细介绍各部分的设计。

\section{页缓存及数据预取}
页缓存(Page Cache)是一种计算机系统常用的用于加速数据访问的技术。利用应用程序访问数据的局部性特点,
通过将常用的文件的数据缓存至内存,从而加速对文件数据的访问。数据预取(Readahead)则是一种预测性的技术,
它根据应用过去的访问模式来预测未来可能需要的数据,并提前从磁盘上读取这些数据到页缓存中。当这些数据
真正被访问时,它们已经在内存中可用,从而避免了磁盘访问的延迟。
因此,对于有明显访问模式的应用(如顺序访问),数据预取可以显著提高应用程序的性能。
本文主要关注顺序访问模式,为Asterinas的页缓存实现了数据预取。

% vim:ts=4:sw=4
