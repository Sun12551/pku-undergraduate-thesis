% Copyright (c) 2014,2016,2021 Casper Ti. Vector
% Public domain.

\begin{cabstract}
	%\pkuthssffaq % 中文测试文字
	在当今数字化时代,信息安全至关重要,而操作系统的安全性又是其中的代表。
	Rust 语言恰以其独特的所有权系统和内存安全等特性著称,为操作系统开发带来了新的可能性,
	能够有效地防范诸如空指针解引用、缓冲区溢出等常见的安全漏洞。
	中关村实验室推出了新型操作系统 Asterinas,其采用 Rust 语言开发,旨在为用户提供更加安全可靠的操作系统环境。
	
	在这一背景下,本研究着眼于 Asterinas 操作系统的文件系统部分,旨在进一步完善其功能。
	我们为 Asterinas 引入 exFAT 文件系统的支持,并对文件系统的页缓存模块增加了数据预取机制。
	此外,我们为新增的 exFAT 文件系统设计了单元测试并运行了集成测试以验证实现的正确性。
	随后,我们针对新增的 exFAT 文件系统和页缓存的数据预取机制,使用 FIO 进行了性能测试,并与 Linux 中的实现进行了比较。
	测试结果表明,使用页缓存时,我们的实现在顺序读写模式上仍然与 Linux 中实现有一定的差距。
	不使用页缓存时,Asterinas 直接与块设备交互的性能要弱于 Linux。
	这个结果说明,页缓存的设计仍有改进的空间,对块设备实现的性能还有待提高,这也将是我们后续继续改进的方向。
	
	
\end{cabstract}

\ifblind\begin{beabstract}\else\begin{eabstract}\fi
	In today's digital age, ensuring information security is crucial.
	And operating system security plays an important role in information security. 
	Rust language is renowned for its unique ownership system and memory safety features, 
	offering new possibilities in operating system development by effectively preventing 
	common security vulnerabilities such as null pointer dereferencing and buffer overflows. 
	The emergence of the new Asterinas operating system from the Zhongguancun Laboratory, 
	which is developed using Rust language, aims to provide users with a more secure and 
	reliable operating system environment.
	
	This study focuses on enhancing the implementation of file system component of the Asterinas operating system. 
	We have introduced support for the exFAT file system in Asterinas and 
	added a data prefetching mechanism to the file system's page cache module. 
	Furthermore, we designed unit tests for the newly added exFAT file system and 
	conducted integration tests to verify the correctness of the implementation.	
	Subsequently, we conducted performance tests using FIO on the newly added exFAT file system 
	and page cache data prefetching mechanism, comparing the results with implementations in Linux.
	The test results indicate that when using the page cache, 
	our implementation still shows some gap compared to Linux in sequential IO mode.
	When not using page caching, Asterinas' performance in directly interacting with block devices is weaker than Linux.
	These results suggest that there is room for improvement in the design of page cache and block device, 
	which will be the focus of future improvements.
\ifblind\end{beabstract}\else\end{eabstract}\fi

% vim:ts=4:sw=4
