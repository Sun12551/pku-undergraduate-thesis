% Copyright (c) 2014,2016,2018 Casper Ti. Vector
% Public domain.

\chapter{引言}

\section{操作系统安全性的重要性}
信息安全是当今数字化社会中至关重要的一个领域,关系到保护数据、系统和网络免受未经授权的访问、使用、泄露、破坏或干扰等方方面面。\parencite{anderson2001,stallings2017,schneier2015,bishop2003,ross2014}
在信息安全领域,操作系统的安全性是一个核心议题,因为操作系统作为计算机系统的基础软件,直接影响整个系统的安全性。\parencite{zhang2020enhancing,wang2019survey,patel2018mitigating}

就像著名的计算机科学家安德鲁·坦宁斯(Andrew Tanenbaum)在其著作《现代操作系统》\parencite{tanenbaum2014modern}中指出:“操作系统的安全性是整个系统安全的基石。” 
但传统的操作系统,尤其是基于 C/C++ 的操作系统,由于其对内存管理等方面的处理特性,或多或少地存在着一定的安全隐患。
这些安全隐患可能被恶意用户或恶意软件利用,从而导致系统遭受攻击,出现数据泄露或服务中断等问题。下面列举了一些常见隐患和漏洞:
\begin{itemize}
    \item 内存管理漏洞: 传统操作系统中常见的 C/C++ 内存管理问题包括缓冲区溢出、空指针解引用和内存泄漏等,这些问题可能导致系统遭受各种类型的攻击。\parencite{example1}
    \item 提权漏洞: 基于 C/C++ 的操作系统中存在提权漏洞,恶意软件或攻击者可以利用这些漏洞获取系统特权,从而执行恶意操作。\parencite{example2}
    \item 代码注入漏洞: 操作系统中的代码注入漏洞可能导致恶意代码被注入到系统进程中,从而破坏系统的完整性和安全性。\parencite{example3}
    \item 过度权限: 传统操作系统中的一些进程可能被赋予过多的权限,这可能导致系统受到内部威胁或恶意行为的影响。\parencite{example4}
\end{itemize}

这些安全隐患的存在凸显了加强操作系统安全性的紧迫性和重要性,需要采取有效的措施来缓解这些问题,以确保系统的安全性和稳定性。因此,如何提高操作系统的安全性,是当今信息科学的重要研究课题。

\section{Rust 语言在操作系统开发中的优势}
Rust 语言\parencite{matsakis2014rust}作为一种系统编程语言,旨在实现“安全、并发、实用”的设计目标。
Rust 语言在操作系统开发中具有诸多优势,包括内存安全保证、优雅的错误处理机制、并发性能、线程安全性和跨平台支持。
这些优势使得 Rust 成为系统开发的一个理想的选择,能够提高操作系统的安全性、可靠性和开发效率。

Rust 语言在操作系统开发中的优势主要体现在以下几个方面:
\begin{itemize}
    \item 内存安全保证:Rust 的所有权系统能够在编译时检测并预防内存安全问题,如空指针解引用和缓冲区溢出等常见内存错误。
    相比于 C/C++ 在这些问题上需要开发者自行处理,Rust 语言的静态检查能够自动发现并解决这些问题,从而显著提高了代码的安全性。
    \item 优雅的错误处理机制:Rust 利用模式匹配、枚举类型等特性简化了对函数返回值的处理,使得程序中的错误处理逻辑更加优雅。
    这种简化不仅减轻了开发者的负担,还有助于提高代码的可读性和可维护性,从而提升了开发效率,在操作系统的开发中更是显得尤为重要。
    \item 并发性能:Rust 的所有权系统和借用检查器使得编写并发代码更加容易和安全,避免常见的并发陷阱,如数据竞争和死锁,提高操作系统的性能和稳定性。
    \item 线程安全性:通过所有权和借用规则,Rust 在编译时避免多线程编程中常见的问题,如数据竞争和内存泄漏,确保操作系统的稳定性和可靠性。
    \item 跨平台支持:Rust 的跨平台支持使得开发者能够轻松编写可移植的操作系统代码,无需担心不同平台之间的兼容性问题,提高操作系统的灵活性和可移植性。
\end{itemize}

因此,Rust 语言通过其独特的特性和优势为操作系统开发带来了显著的益处。
Rust 通过其内存安全保证和优雅的错误处理机制,为操作系统开发提供了一种更加安全、高效的编程方式。
同时,Rust 语言的高性能特性和高开发效率也使其成为开发操作系统的理想选择,
能够在保证系统性能的同时提升开发效率,为构建更安全、更可靠的操作系统奠定坚实基础。

\section{本研究的主要内容和目标}
本研究立足于使用 Rust 编写的新兴操作系统 Asterinas\parencite{Asterinas},旨在完善其文件系统实现。
我们为 Asterinas 新增了 exFAT 文件系统\parencite{exFAT,exFAT1,exFAT2}的实现,并利用 Rust 语言的特性进行定制化设计。此外,
本研究还为其文件系统的页缓存\parencite{tanenbaum2014modern,ostep,bovet2005understanding}增加了
数据预取\parencite{readahead,sequential_prefetching}的功能,可以很大程度上改善页缓存在顺序读写模式下的
性能表现,有助于进一步提高文件系统的性能,这一点也在测试中得到了证实。
随后,为了验证实现的正确性,我们针对新增的 exFAT 文件系统编写了单元测试并进行了集成测试。
最后,使用文件 IO 测试工具 FIO 对新增的 exFAT 文件系统进行了性能测试。
通过与 Linux 中的对应实现进行比较,我们分析了现有实现的不足并提出未来可能的改进方向。

本文剩余部分按下述结构进行组织:
\begin{itemize}
    \item 第二章介绍本工作的相关背景,包括 Rust 语言、exFAT 文件系统和页缓存数据预取。
    \item 第三章阐述 exFAT 文件系统的具体设计实现。
    \item 第四章阐述页缓存中数据预取的具体设计实现。
    \item 第五章针对实现的 exFAT 文件系统进行了正确性测试,主要包括单元测试和集成测试。
    \item 第六章针对实现的 exFAT 文件系统进行了性能测试,并与 Linux 中的实现进行了比较。
    \item 第七章对文章进行总结并提出未来可能的工作方向。
\end{itemize}



%\newpage
%偶数页的风格是这样的。

% vim:ts=4:sw=4
