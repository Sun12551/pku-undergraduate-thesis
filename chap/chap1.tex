% Copyright (c) 2014,2016,2018 Casper Ti. Vector
% Public domain.

\chapter{引言}

\section{操作系统安全性的重要性}
在当今的信息化社会,操作系统作为计算机系统的核心,其安全性直接影响到整个计算机系统的安全,
以及用户数据的安全。然而,传统的操作系统,如基于C/C++的操作系统,由于其内存管理等特性,
存在一定的安全隐患。因此,如何提高操作系统的安全性,是当前计算机科学领域的重要研究课题。

\section{Rust语言在操作系统开发中的优势}
Rust语言是一种系统编程语言,其设计目标是“安全、并发、实用”。
Rust语言的最大特点就是其所有权系统,这个系统保证了在编译时期就可以检查出内存安全问题,
包括空指针解引用、缓冲区溢出等常见的内存错误,这些在C/C++中需要开发者自行处理的问题,
Rust语言可以在编译阶段自动检查并处理,大大提高了代码的安全性。
同时,Rust利用自身的enum类型等特性,简化了对函数返回值的检查,使得程序中的错误处理逻辑更为优雅,
减轻了开发者的负担,有助于提高开发效率。
因此,使用Rust语言开发操作系统,有望提高操作系统的安全性和可靠性,同时保持高性能和高效的开发效率。

\section{本研究的目标和重要性}
本研究立足于使用Rust编写的新兴操作系统Asterinas\parencite{Asterinas},旨在完善其文件系统实现,
我们为Asterinas新增了exFAT文件系统\parencite{exFAT}的实现,并利用Rust语言的特性进行定制化设计。此外,
本研究还为其文件系统的页缓存增加了数据预取的功能,可以很大程度上改善页缓存在顺序读写模式下的
性能表现,有助于进一步提高文件系统的性能,这一点也在测试中得到了证实。

本文剩余部分按下述结构进行组织:第二章介绍exFAT文件系统和页缓存数据预取的相关背景;第三章阐述exFAT文件
系统的具体设计;第四章阐述页缓存数据预取的具体设计;第五章针对exFAT文件系统和页缓存性能进行了测试,
并与Linux进行了比较;第六章对文章进行总结并提出未来可能的工作方向。


%\newpage
%偶数页的风格是这样的。

% vim:ts=4:sw=4
